\documentclass[aspectratio=169]{beamer}
\usepackage{beamertheme}

% Presentation metadata
\title{Wasserwirtschaft}
\subtitle{Kapitel 4 Hochwasser – Einführung: Beispiel Wieslauf}
\author{Sebastian Schwindt}
\date{\\today}

\begin{document}

% Title slide
\maketitle

% Slide 2
\begin{frame}{Hochwasser}
\framesubtitle{Beispiel: Extremhochwasser Wieslauf \& Rems im Juni 2024}
\begin{itemize}
  \item {\small \textit{Bildquelle: OpenStreetMap contributors / Tracestrack.  (2025)}}
  \item Wieslauf
  \item Rems
  \item Schorndorf
  \item \textbf{Rudersberg}
\end{itemize}
\begin{center}
  \includegraphics[width=0.9\textwidth]{slide02_img001.jpg}
\end{center}
\end{frame}

% Slide 3
\begin{frame}{Hochwasser}
\framesubtitle{Beispiel: Extremhochwasser Wieslauf \& Rems im Juni 2024}
\begin{itemize}
  \item {\small \textit{Bildquelle: }}{\small \textit{https://www.hvz.baden-wuerttemberg.de/pegel.html?id=00248}}{\small \textit{ }}
\end{itemize}
\begin{center}
  \includegraphics[width=0.9\textwidth]{slide03_img002.jpg}
\end{center}
\begin{center}
  \includegraphics[width=0.49\textwidth]{slide03_img003.jpg}
\end{center}
\end{frame}

% Slide 4
\begin{frame}{Hochwasser}
\framesubtitle{Beispiel: Extremhochwasser Wieslauf \& Rems im Juni 2024}
\begin{itemize}
  \item {\small \textit{Datenquelle: }}{\small \textit{https://www.hvz.baden-wuerttemberg.de/pegel.html?id=00248}}{\small \textit{ }}
  \item {\footnotesize \textbf{1. Hochwasserkennwerte}}
  \item {\footnotesize Im konkreten Bemessungsfall müssen umfassendere Betrachtungen erfolgen!}
  \item {\footnotesize Informieren Sie sich zusätzlich, ob Hochwassergefahrenkarten für diesen Bereich vorliegen.}
  \item {\footnotesize \textbf{1.1 Hochwasserabfluss}}
  \item {\footnotesize \textbf{100-jährlicher Hochwasserabfluss }}\textcolor[RGB]{127,0,127}{{\footnotesize \textbf{HQ 100: 77.9 m³/s}}}{\footnotesize \textbf{ }}{\footnotesize Quelle: Regionalisierung (Stand: 03.12.2013)}
  \item {\footnotesize 50-jährlicher Hochwasserabfluss }\textcolor[RGB]{255,0,255}{{\footnotesize \textbf{HQ 50: 67.0 m³/s}}}
  \item {\footnotesize 20-jährlicher Hochwasserabfluss HQ 20: 53.4 m³/s}
  \item {\footnotesize 10-jährlicher Hochwasserabfluss HQ 10: 43.6 m³/s}
  \item {\footnotesize 5-jährlicher Hochwasserabfluss   HQ 5:   34.0 m³/s}
  \item {\footnotesize 2-jährlicher Hochwasserabfluss   HQ 2:   20.9 m³/s}
  \item {\footnotesize \textbf{1.2 Hochwasserstand}}
  \item {\footnotesize \textbf{100-jährlicher Hochwasserstand HW 100: 3.17 m }}{\footnotesize Quelle: berechnet aus HQ 100 und aktueller WQ-Beziehung}
  \item {\footnotesize 50-jährlicher Hochwasserstand HW 50: 2.89 m}
  \item {\footnotesize 20-jährlicher Hochwasserstand HW 20: 2.54 m}
  \item {\footnotesize 10-jährlicher Hochwasserstand HW 10: 2.30 m}
  \item {\footnotesize 5-jährlicher Hochwasserstand   HW 5:   2.04 m}
  \item {\footnotesize 2-jährlicher Hochwasserstand   HW 2:   1.58 m}
  \item {\footnotesize \textbf{2. Mittelwasserkennwerte}}
  \item {\footnotesize Mittelwert Abfluss MQ:0.92 m³/s Quelle: Regionalisierung (Stand: 01.03.2016)}
  \item {\footnotesize Mittelwert Wasserstand MW: 0.33 m Quelle: berechnet aus MQ und aktueller WQ-Beziehung}
  \item {\footnotesize \textbf{3. Niedrigwasserkennwerte}}
  \item {\footnotesize Mittelwert niedrigster jährlicher Abflüsse MNQ: 0.15 m³/s Quelle: Regionalisierung (Stand: 01.03.2016)}
  \item {\footnotesize Mittelwert niedrigster jährlicher Wasserstände MNW:0.15m Quelle: berechnet aus MNQ und aktueller WQ-Beziehung}
  \item {\footnotesize Niedrigster Abfluss im Zeitraum 1981-2023: 12.08.1998 NQ:0.04 m³/s Quelle: Tagesmittelwerte}
  \item {\footnotesize Niedrigster Wasserstand NW: 0.07 m Quelle: Berechnet aus NQ und aktueller WQ-Beziehung}
\end{itemize}
\end{frame}

% Slide 5
\begin{frame}{Hochwasser}
\framesubtitle{Beispiel: Extremhochwasser Wieslauf \& Rems im Juni 2024}
\begin{itemize}
  \item {\small \textit{Bildquelle: }}{\small \textit{https://www.hvz.baden-wuerttemberg.de/pegel.html?id=00248}}{\small \textit{ }}
\end{itemize}
\begin{center}
  \includegraphics[width=0.73\textwidth]{slide05_img004.jpg}
\end{center}

% AutoShape converted to TikZ
\begin{tikzpicture}[overlay, remember picture]
  \node[draw, fill=yellow!30, rounded corners, drop shadow, text width=6.5cm, align=center] at ([xshift=0.49\textwidth, yshift=3.8cm]current page.south west) {Ca. 155 m³/s !};
\end{tikzpicture}

% AutoShape converted to TikZ
\begin{tikzpicture}[overlay, remember picture]
  \node[draw, fill=blue!10, text width=0.4cm, align=center] at ([xshift=0.22\textwidth, yshift=0.7cm]current page.south west) {24};
\end{tikzpicture}

% AutoShape converted to TikZ
\begin{tikzpicture}[overlay, remember picture]
  \node[draw, fill=blue!10, text width=0.4cm, align=center] at ([xshift=0.28\textwidth, yshift=0.7cm]current page.south west) {24};
\end{tikzpicture}

% AutoShape converted to TikZ
\begin{tikzpicture}[overlay, remember picture]
  \node[draw, fill=blue!10, text width=0.4cm, align=center] at ([xshift=0.32\textwidth, yshift=0.7cm]current page.south west) {24};
\end{tikzpicture}

% AutoShape converted to TikZ
\begin{tikzpicture}[overlay, remember picture]
  \node[draw, fill=blue!10, text width=0.4cm, align=center] at ([xshift=0.37\textwidth, yshift=0.7cm]current page.south west) {24};
\end{tikzpicture}

% AutoShape converted to TikZ
\begin{tikzpicture}[overlay, remember picture]
  \node[draw, fill=blue!10, text width=0.4cm, align=center] at ([xshift=0.42\textwidth, yshift=0.7cm]current page.south west) {24};
\end{tikzpicture}

% AutoShape converted to TikZ
\begin{tikzpicture}[overlay, remember picture]
  \node[draw, fill=blue!10, text width=0.4cm, align=center] at ([xshift=0.47\textwidth, yshift=0.7cm]current page.south west) {24};
\end{tikzpicture}

% AutoShape converted to TikZ
\begin{tikzpicture}[overlay, remember picture]
  \node[draw, fill=blue!10, text width=0.4cm, align=center] at ([xshift=0.52\textwidth, yshift=0.7cm]current page.south west) {24};
\end{tikzpicture}

% AutoShape converted to TikZ
\begin{tikzpicture}[overlay, remember picture]
  \node[draw, fill=blue!10, text width=0.4cm, align=center] at ([xshift=0.58\textwidth, yshift=0.7cm]current page.south west) {24};
\end{tikzpicture}

% AutoShape converted to TikZ
\begin{tikzpicture}[overlay, remember picture]
  \node[draw, fill=blue!10, text width=0.4cm, align=center] at ([xshift=0.63\textwidth, yshift=0.7cm]current page.south west) {24};
\end{tikzpicture}

% AutoShape converted to TikZ
\begin{tikzpicture}[overlay, remember picture]
  \node[draw, fill=blue!10, text width=0.4cm, align=center] at ([xshift=0.67\textwidth, yshift=0.7cm]current page.south west) {24};
\end{tikzpicture}

% AutoShape converted to TikZ
\begin{tikzpicture}[overlay, remember picture]
  \node[draw, fill=blue!10, text width=0.4cm, align=center] at ([xshift=0.73\textwidth, yshift=0.7cm]current page.south west) {24};
\end{tikzpicture}

% AutoShape converted to TikZ
\begin{tikzpicture}[overlay, remember picture]
  \node[draw, fill=blue!10, text width=0.4cm, align=center] at ([xshift=0.78\textwidth, yshift=0.7cm]current page.south west) {24};
\end{tikzpicture}

% AutoShape converted to TikZ
\begin{tikzpicture}[overlay, remember picture]
  \node[draw, fill=blue!10, text width=1.8cm, align=center] at ([xshift=0.26\textwidth, yshift=0.5cm]current page.south west) {01.01.2024};
\end{tikzpicture}

% AutoShape converted to TikZ
\begin{tikzpicture}[overlay, remember picture]
  \node[draw, fill=blue!10, text width=1.8cm, align=center] at ([xshift=0.40\textwidth, yshift=0.5cm]current page.south west) {31.12.2024};
\end{tikzpicture}
\end{frame}

% Slide 6
\begin{frame}{Hochwasser}
\framesubtitle{Beispiel: Extremhochwasser Wieslauf \& Rems im Juni 2024}
\begin{itemize}
  \item {\small \textit{Videoquelle: YouTube / Bebop D.}}
\end{itemize}
% Video: slide06_vid001.mp4
\includemovie[
    attach=false,
    autoplay,
    text={\includegraphics[width=\textwidth]{videos/generic-thumbnail.jpg}}
]{\textwidth}{\textheight}{videos/slide06_vid001.mp4}
\end{frame}

% Slide 7
\begin{frame}{Hochwasser}
\framesubtitle{Beispiel: Extremhochwasser Wieslauf \& Rems im Juni 2024}
\begin{center}
  \includegraphics[width=0.71\textwidth]{slide07_img005.jpg}
\end{center}

% AutoShape converted to TikZ
\begin{tikzpicture}[overlay, remember picture]
  \node[draw, fill=blue!10, text width=5.6cm, align=center] at ([xshift=0.70\textwidth, yshift=3.4cm]current page.south west) {Schaden: ca. € 120 Mio.};
\end{tikzpicture}
\end{frame}

% Slide 8
\begin{frame}{Hochwasser}
\framesubtitle{Beispiel: Extremhochwasser Wieslauf \& Rems im Juni 2024}
\begin{center}
  \includegraphics[width=0.9\textwidth]{slide08_img006.jpg}
\end{center}
\end{frame}

% Slide 9
\begin{frame}{Hochwasser}
\framesubtitle{Beispiel: Extremhochwasser Wieslauf \& Rems im Juni 2024}
\begin{itemize}
  \item {\Large Renaturierung bei Winterbach im April 2020}
  \item {\Large Juni 2024}
\end{itemize}
\begin{center}
  \includegraphics[width=0.66\textwidth]{slide09_img007.jpg}
\end{center}
\begin{center}
  \includegraphics[width=0.66\textwidth]{slide09_img008.jpg}
\end{center}
\end{frame}

% Thank you slide
\thankyou{Thank you for your attention!}{Sebastian Schwindt}{Position}{email@example.com}{theme/logos/drop.png}

\end{document}