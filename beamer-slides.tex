\documentclass[aspectratio=169]{beamer}
\usepackage{beamertheme}

% Presentation metadata
\title{Wasserwirtschaft}
\subtitle{Kapitel 4 Hochwasser – Beispiel Wieslauf}
\author{Sebastian Schwindt}
\date{\\today}

\begin{document}

% Title slide
\maketitle

% Slide 2
\begin{frame}{Hochwasser}
\framesubtitle{Beispiel: Extremhochwasser Wieslauf \& Rems im Juni 2024}
\begin{itemize}
  \item {\small \textit{Bildquelle: }}{\small \textit{OpenStreetMap contributors / }}{\small \textit{Tracestrack.  (2025)}}
  \item Wieslauf
  \item Rems
  \item Schorndorf
  \item \textbf{Rudersberg}
\end{itemize}
\begin{center}
  \includegraphics[width=0.9\textwidth]{slide02_img001.jpg}
\end{center}
\end{frame}

% Slide 3
\begin{frame}{Hochwasser}
\framesubtitle{Beispiel: Extremhochwasser Wieslauf \& Rems im Juni 2024}
\begin{itemize}
  \item {\small \textit{Bildquelle: }}{\small \textit{https://www.hvz.baden-wuerttemberg.de/}}{\small \textit{pegel.html?id=00248}}{\small \textit{ }}
\end{itemize}
\begin{center}
  \includegraphics[width=0.9\textwidth]{slide03_img002.jpg}
\end{center}
\begin{center}
  \includegraphics[width=0.49\textwidth]{slide03_img003.jpg}
\end{center}
\end{frame}

% Slide 4
\begin{frame}{Hochwasser}
\framesubtitle{Beispiel: Extremhochwasser Wieslauf \& Rems im Juni 2024}
\begin{itemize}
  \item {\small \textit{Datenquelle: }}{\small \textit{https://www.hvz.baden-wuerttemberg.de/}}{\small \textit{pegel.html?id=00248}}{\small \textit{ }}
  \item {\footnotesize 1. }{\footnotesize Hochwasserkennwerte}
  \item {\footnotesize Im }{\footnotesize konkreten }{\footnotesize Bemessungsfall }{\footnotesize müssen }{\footnotesize umfassendere }{\footnotesize Betrachtungen }{\footnotesize erfolgen!}
  \item {\footnotesize Informieren }{\footnotesize Sie }{\footnotesize sich }{\footnotesize zusätzlich, }{\footnotesize ob }{\footnotesize Hochwassergefahrenkarten }{\footnotesize für }{\footnotesize diesen }{\footnotesize Bereich }{\footnotesize vorliegen.}
  \item {\footnotesize 1.1 }{\footnotesize Hochwasserabfluss}
  \item {\footnotesize \textbf{100-jährlicher }}{\footnotesize \textbf{HochwasserabflussHQ 100: }}{\footnotesize \textbf{77.9m³/}}{\footnotesize \textbf{sQuelle: }}{\footnotesize \textbf{Regionalisierung (Stand: 03.12.2013)}}
  \item {\footnotesize 50-jährlicher }{\footnotesize HochwasserabflussHQ 50:}{\footnotesize 67.0m³/}{\footnotesize sQuelle: }{\footnotesize Regionalisierung (Stand: 03.12.2013)}
  \item {\footnotesize 20-jährlicher }{\footnotesize HochwasserabflussHQ 20:}{\footnotesize 53.4m³/}{\footnotesize sQuelle: }{\footnotesize Regionalisierung (Stand: 03.12.2013)}
  \item {\footnotesize 10-jährlicher }{\footnotesize HochwasserabflussHQ 10:}{\footnotesize 43.6m³/}{\footnotesize sQuelle: }{\footnotesize Regionalisierung (Stand: 03.12.2013)}
  \item {\footnotesize 5-jährlicher }{\footnotesize HochwasserabflussHQ 5:}{\footnotesize 34.0m³/}{\footnotesize sQuelle: }{\footnotesize Regionalisierung (Stand: 03.12.2013)}
  \item {\footnotesize 2-jährlicher }{\footnotesize HochwasserabflussHQ 2:}{\footnotesize 20.9m³/}{\footnotesize sQuelle: }{\footnotesize Regionalisierung (Stand: 03.12.2013)}
  \item {\footnotesize 1.2 }{\footnotesize Hochwasserstand}
  \item {\footnotesize \textbf{100-jährlicher }}{\footnotesize \textbf{HochwasserstandHW 100: }}{\footnotesize \textbf{3.17mQuelle: }}{\footnotesize \textbf{berechnet }}{\footnotesize \textbf{aus }}{\footnotesize \textbf{HQ 100 }}{\footnotesize \textbf{und }}{\footnotesize \textbf{aktueller }}{\footnotesize \textbf{WQ-Beziehung}}
  \item {\footnotesize 50-jährlicher }{\footnotesize HochwasserstandHW 50:}{\footnotesize 2.89mQuelle: }{\footnotesize berechnet }{\footnotesize aus }{\footnotesize HQ 50 }{\footnotesize und }{\footnotesize aktueller }{\footnotesize WQ-Beziehung}
  \item {\footnotesize 20-jährlicher }{\footnotesize HochwasserstandHW 20:}{\footnotesize 2.54mQuelle: }{\footnotesize berechnet }{\footnotesize aus }{\footnotesize HQ 20 }{\footnotesize und }{\footnotesize aktueller }{\footnotesize WQ-Beziehung}
  \item {\footnotesize 10-jährlicher }{\footnotesize HochwasserstandHW 10:}{\footnotesize 2.30mQuelle: }{\footnotesize berechnet }{\footnotesize aus }{\footnotesize HQ 10 }{\footnotesize und }{\footnotesize aktueller }{\footnotesize WQ-Beziehung}
  \item {\footnotesize 5-jährlicher }{\footnotesize HochwasserstandHW 5:}{\footnotesize 2.04mQuelle: }{\footnotesize berechnet }{\footnotesize aus }{\footnotesize HQ 5 }{\footnotesize und }{\footnotesize aktueller }{\footnotesize WQ-Beziehung}
  \item {\footnotesize 2-jährlicher }{\footnotesize HochwasserstandHW 2:}{\footnotesize 1.58mQuelle: }{\footnotesize berechnet }{\footnotesize aus }{\footnotesize HQ 2 }{\footnotesize und }{\footnotesize aktueller }{\footnotesize WQ-Beziehung}
  \item {\footnotesize 2. }{\footnotesize Mittelwasserkennwerte}
  \item {\footnotesize Mittelwert }{\footnotesize AbflussMQ:}{\footnotesize 0.92m³/}{\footnotesize sQuelle: }{\footnotesize Regionalisierung (Stand: 01.03.2016)}
  \item {\footnotesize Mittelwert }{\footnotesize WasserstandMW:}{\footnotesize 0.33mQuelle: }{\footnotesize berechnet }{\footnotesize aus }{\footnotesize MQ }{\footnotesize und }{\footnotesize aktueller }{\footnotesize WQ-Beziehung}
  \item {\footnotesize 3. }{\footnotesize Niedrigwasserkennwerte}
  \item {\footnotesize Mittelwert }{\footnotesize niedrigster }{\footnotesize jährlicher }{\footnotesize AbflüsseMNQ:}{\footnotesize 0.15m³/}{\footnotesize sQuelle: }{\footnotesize Regionalisierung (Stand: 01.03.2016)}
  \item {\footnotesize Mittelwert }{\footnotesize niedrigster }{\footnotesize jährlicher }{\footnotesize WasserständeMNW:}{\footnotesize 0.15mQuelle: }{\footnotesize berechnet }{\footnotesize aus }{\footnotesize MNQ }{\footnotesize und }{\footnotesize aktueller }{\footnotesize WQ-Beziehung}
  \item {\footnotesize Niedrigster }{\footnotesize Abfluss }{\footnotesize im }{\footnotesize Zeitraum 1981-2023: }{\footnotesize 12.08.1998NQ:}{\footnotesize 0.04m³/}{\footnotesize sQuelle: }{\footnotesize Tagesmittelwerte}
  \item {\footnotesize Niedrigster }{\footnotesize Wasserstand NW:}{\footnotesize 0.07mQuelle: }{\footnotesize Berechnet }{\footnotesize aus }{\footnotesize NQ }{\footnotesize und }{\footnotesize aktueller }{\footnotesize WQ-Beziehung}
\end{itemize}
\end{frame}

% Slide 5
\begin{frame}{Hochwasser}
\framesubtitle{Beispiel: Extremhochwasser Wieslauf \& Rems im Juni 2024}
\begin{itemize}
  \item {\small \textit{Bildquelle: }}{\small \textit{https://www.hvz.baden-wuerttemberg.de/}}{\small \textit{pegel.html?id=00248}}{\small \textit{ }}
\end{itemize}
\begin{center}
  \includegraphics[width=0.73\textwidth]{slide05_img004.jpg}
\end{center}

% AutoShape converted to TikZ
\begin{tikzpicture}[overlay, remember picture]
  \node[draw, fill=yellow!30, rounded corners, drop shadow, text width=6.5cm, align=center] at ([xshift=0.49\textwidth, yshift=3.8cm]current page.south west) {Ca. 155 m³/s !};
\end{tikzpicture}
\end{frame}

% Slide 6
\begin{frame}{Hochwasser}
\framesubtitle{Beispiel: Extremhochwasser Wieslauf \& Rems im Juni 2024}
\begin{itemize}
  \item {\small \textit{Videoquelle: YouTube / Bebop D.}}
\end{itemize}
\begin{center}
  \includegraphics[width=0.87\textwidth]{slide06_img005.jpg}
\end{center}
\end{frame}

% Slide 7
\begin{frame}{Hochwasser}
\framesubtitle{Beispiel: Extremhochwasser Wieslauf \& Rems im Juni 2024}
\begin{itemize}
  \item {\small \textit{Bildquelle: }}{\small \textit{zeit.de}}
\end{itemize}

% AutoShape converted to TikZ
\begin{tikzpicture}[overlay, remember picture]
  \node[draw, fill=blue!10, text width=5.6cm, align=center] at ([xshift=0.71\textwidth, yshift=2.7cm]current page.south west) {Schaden: ca. € 120 Mio.};
\end{tikzpicture}
\end{frame}

% Slide 8
\begin{frame}{Hochwasser}
\framesubtitle{Beispiel: Extremhochwasser Wieslauf \& Rems im Juni 2024}
\begin{center}
  \includegraphics[width=0.71\textwidth]{slide08_img006.jpg}
\end{center}
\end{frame}

% Slide 9
\begin{frame}{Hochwasser}
\framesubtitle{Beispiel: Extremhochwasser Wieslauf \& Rems im Juni 2024}
\begin{center}
  \includegraphics[width=0.9\textwidth]{slide09_img007.jpg}
\end{center}
\end{frame}

% Slide 10
\begin{frame}{Hochwasser}
\framesubtitle{Beispiel: Extremhochwasser Wieslauf \& Rems im Juni 2024}
\begin{itemize}
  \item {\Large Renaturierung }{\Large bei }{\Large Winterbach }{\Large im April 2020}
  \item {\Large Juni 2024}
\end{itemize}
\begin{center}
  \includegraphics[width=0.66\textwidth]{slide10_img008.jpg}
\end{center}
\begin{center}
  \includegraphics[width=0.66\textwidth]{slide10_img009.jpg}
\end{center}
\end{frame}

% Thank you slide
\thankyou{Thank you for your attention!}{Sebastian Schwindt}{Position}{email@example.com}{theme/logos/drop.png}

\end{document}